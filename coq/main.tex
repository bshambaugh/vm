\documentclass{article}
\usepackage[utf8]{inputenc}

\usepackage{stmaryrd}
\usepackage[left=1.1in,right=1.1in,top=1in,bottom=1in]{geometry}
%\usepackage{xypic}
\usepackage{mathpartir}
%\usepackage{irule}
\usepackage{indentfirst}
\usepackage{amssymb}
\usepackage{amsmath}
\usepackage{graphicx}
\usepackage{hyperref}
%\usepackage{multicol}
%\setlength{\columnsep}{.25in}
\renewcommand{\baselinestretch}{1.1}
%\setlength{\parskip}{20pt}
\def\taking{:}

\title{Stream Ring Theory, Correctly}
\author{Marko and Ryan}
\date{\today}


\begin{document}

\maketitle
\abstract{In this paper we correct the mathematical errors in~\cite{srt}, resulting in a theory similar in spirit.  All our proofs are formalized in the Coq proof assistant~\cite{coq}.  Familiarity with category theory at the level of~\cite{ct} is required to understand this paper.  This paper is a work in progress; we have only formalized the first few sections of~\cite{srt}.}

\section{Bags and Streams}
\indent 
Let $X$ be a set.  A {\it bag}~\cite{bags} (or {\it multi-set}) over $X$ is a possibly empty list (ordered sequence) $[x_1,\ldots,x_n]$, with every $x$ in $X$, considered up to permutation.  In this paper, we refer to the underlying list as the bag's {\it stream}, to each $x$ as a {\it stream object}, to the set of all bags over $X$ as ${\sf Bag}(X)$, and for every set $X$, to every function of the form $f : Y \to {\sf Bag}(X)$ as a {\it stream function} from $Y$ to $X$.  

As examples, the stream function {\sf cons} $: X \times {\sf Bag}(X) \to {\sf Bag}(X)$ inserts a stream object into a bag, and the stream function {\sf remove} : $X \times {\sf Bag}(X) \to {\sf Bag}(X)$ removes an object from a bag, if present, otherwise returning its input. 


As a technical convenience, and without loss of generality, in this paper we will assume that every set $X$ is equipped with a total order $\preceq_X$, and that every bag's stream is sorted with respect to $\preceq_X$.  This assumption allows us to define a unique representative list for each bag, facilitating the Coq proofs, and follows from the axiom of choice and classical logic. 


{\bf Lemma 1}. The stream functions form a category $\mathcal{F}$ (depending on $\preceq$) as follows: 

\begin{itemize}
    \item The objects of $\mathcal{F}$ are all the sets. 
    \item The arrows of $\mathcal{F}$ from $X$ to $Y$ are all the stream functions from $X$ to $Y$.
    \item Composition of $f : X \to {\sf Bag}(Y)$ and $g : Y \to {\sf Bag}(Z)$ given by: 
$$(f;g)(x \in X) \in {\sf Bag}(Z) \ := \  \biguplus_{y \in f(x)} g(y),$$ where $\uplus$ indicates bag union (appending and sorting the underlying streams). 
\item The identity ${\sf id}_X : X \to {\sf Bag}(X)$ is the function taking each $x \in X$ to $[x]$.  \hfill $\Box$ 
\end{itemize}  
The next three lemmas prove that $\mathcal{F}$ has all finite products and co-products. 

{\bf Lemma 2}. The empty set is initial and final in $\mathcal{F}$. \hfill $\Box$

{\bf Lemma 3}. For all sets $X$ and $Y$, the disjoint union $X+Y$ and stream functions: $$\iota_1(x \in X) \in {\sf Bag}(X+Y) := [{\sf inl}(x)] \ \ \ \ \iota_2(y \in Y) \in {\sf Bag}(X+Y) := [{\sf inr}(y)]$$ form a coproduct in $\mathcal{F}$, where for every set $Z$ and stream functions: $$i_1 : X \to {\sf Bag}(X+Y) \ \ \ \ i_2 : Y \to {\sf Bag}(X+Y)$$ the case analysis morphism $X+Y \to Z$ sends each $x \in X$ to $i_1(x)$ and each $y \in Y$ to $i_2(y)$. \hfill $\Box$

{\bf Lemma 4}. For all sets $X$ and $Y$, the disjoint union $X+Y$ and stream functions: $$\pi_1(x \in X+Y) \in {\sf Bag}(X) := {\sf if } \ x \in X \ {\sf then} \ [x] \ {\sf else} \ [] \ \ \ \pi_2(x \in X+Y) \in {\sf Bag}(Y) := {\sf if} \ y \in Y \ {\sf then} \ [y] \ {\sf else} \ []$$ form a product in $\mathcal{F}$, where for every set $Z$ and stream functions: $$p_1 : Z \to {\sf Bag}(X) \ \ \ \  p_2 : Z \to {\sf Bag}(Y)$$ the pairing morphism $Z \to X+Y$ sends each $z \in Z$ to ${\sf bmap}({\sf inl}, p_1(z)) \uplus {\sf bmap}({\sf inr}, p_2(z))$, where $${\sf bmap}(f,[v_1, \ldots, v_n]) := [f(v_1)] \uplus \ldots \uplus [f(v_n)]$$ denotes the bag-mapping function. \hfill $\Box$

In this paper, we are primarily interested in the fact that $\mathcal{F}$ is a {\it ringoid}: a category where each hom-set is an abelian group:

{\bf Lemma 4}.  For all sets $X$ and $Y$, the functions $X \to {\sf Bag}(Y)$ form an abelian group, with $(f+g)(x) := f(x) \uplus g(x)$ and $0(x) := []$.  \hfill $\Box$




%\subsection{Extra text}

%In this paper, we will often consider bags where the underlying set $X$ is a cartesian product $X'\times C$, where $C$ is the carrier set for a ring. 

%We will write $l \sim l'$ to indicate two lists $l$ and $l'$ are permutations of each other.
 


\bibliographystyle{plain}
\bibliography{bib}

\section{Appendix: Coq Assumptions}

\begin{verbatim}
zorn : forall T : Type, exists R : relation T, well_order T R

proof_irrelevance : forall (P : Prop) (p1 p2 : P), p1 = p2

functional_extensionality_dep : forall (A : Type) (B : A -> Type)
   (f g : forall x : A, B x),  (forall x : A, f x = g x) -> f = g
   
em : forall P : Prop, {P} + {~ P}

constructive_indefinite_description : forall (A : Type) (P : A -> Prop),
   (exists x : A, P x) -> {x : A | P x}
\end{verbatim}

\end{document}
